\documentclass{article}
\usepackage[utf8]{inputenc}

\title{The Pigeonhole Principle}

\date{}

\begin{document}


1.2 Let \emph{f(n)} be the largest prime divisor of \emph{n}. Can it happen that \emph{x $<$ y} but \emph{f(x) $>$ f(y)}? Give an example or explain why it is impossible. 
\\

Largest divisor by definition is a number that divides another number completely. Prime divisors, in addition is a prime number. 
\\

The example of fact that \emph{x $<$ y} but \emph{f(x) $>$ f(y)} is 
\emph{x = 5 , y = 8 , f(x)=5 , f(y)=2 , x $<$ y but f(x) $>$ f(y) }

\end{document}